For this project, we have two Dockerfiles that create docker images that use the Debian Linux distribution
as the base. The \verb|root-debian| image runs as \verb|root| with root privileges by 
default, while the \verb|non-root-debian| image runs as \verb|appuser| without root privileges.
These two were built using the \verb|docker-compose| in the \verb|app| directory.

We also created Pod Security Policies to enforce specific rules in the Kubernetes cluster.

We will also be looking into the use of network proxies and how it can be used to improve security
of an application. We will specifically be looking into Envoy, which is an open source proxy that is designed
for cloud-based applications/services. There are a number of external services, such as Open Policy Agent,
that Envoy can be customized with that will implement different forms of security measures. The Envoy proxy
configuration is configured using a YAML file and consists of listeners, filters, and clusters. A listener
configures the IP addresses and ports that the proxy will listen to for network requests. With Envoy specifically,
it is run within a Docker container. Filters are defined by \verb|filter_chains|, and handle finding matches with incoming
network request to a destination. A filter can be unique to a listener or can be shared amongst multiple listeners.
After a filter finds the match between request and destination, that request is passed onto a cluster. The cluster
defines the host of the proxy.
