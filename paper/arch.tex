As mentioned above, this experiment is divided into two parts. Part one demonstrates 
how Pod Security Policies can prevent a common vulnerability, and part two demonstrates 
how Envoy can be configured to better fortify the network security of an application.

\subsection{Part 1: Pod Security Policies}
For part one, we will be  demonstrating how Pod Security Policies can prevent privilege 
escalation. We wrote two Dockerfiles that create docker images that use the Debian Linux 
distribution as the base. The \verb|root-debian| image runs as the root user with root 
privileges by default, while the \verb|non-root-debian| image runs as \verb|appuser| 
without root privileges. Both images have a secret text file (\verb|/root/secrets.txt|) 
that only the root user has access to. These two were built using the \verb|docker-compose| 
in the \verb|app| directory. 

We also created two Pod Security Policies to enforce specific rules in the Kubernetes cluster. 
The permissive (\verb|psp-permissive.yaml|) policy allows pods to run as \verb|root|, while 
the restrictive (\verb|psp-restrictive.yaml|) policy does not allow pods to run as \verb|root|, 
and prevents privilege escalation. 

We first deployed both applications into separate pods without Pod Security Policies applied. 
Then we applied the restrictive policy to the cluster and redeployed the applications. 
For each of the deployments, we observed whether or not these distributed applications successfully 
deployed, and if they did, we attempted to escalate privileges to the root user and access the 
secret in \verb|/root/secrets.txt|. Follow the \verb|/psp/README.md| file for a step-by-step 
setup and procedure to reproduce the experiment results detailed in the next section.

\subsection{Part 2: Network Proxy - Envoy}
For part two, we will specifically be looking into Envoy, which is an open source proxy that is 
designed for cloud-based applications/services. There are a number of external services, such as 
Open Policy Agent, that Envoy can be customized with that will implement different forms of 
security measures. 

The Envoy proxy configuration is configured using a YAML file and consists of listeners, filters, 
and clusters. A listener configures the IP addresses and ports that the proxy will listen to for 
network requests. With Envoy specifically, it is run within a Docker container. Filters are 
defined by \verb|filter_chains|, and handle finding matches with incoming network request to a 
destination. A filter can be unique to a listener or can be shared amongst multiple listeners. 
After a filter finds the match between request and destination, that request is passed onto a cluster. 
The cluster defines the host of the proxy. 

We applied an SSL Certificate to an API request and redirect HTTP to an HTTPS request, and utilized 
Envoy's external authorization function alongside Open Policy Agent to define policies for network 
requests that Envoy handles. For applying a SSL Certificate and redirecting network traffic to HTTPS, 
we will be following along the example in "Securing traffic with HTTPS and SSL/TLS" \cite{envoy_ssl}. 
For Envoy's use with Open Policy Agent, we will be following the example given in "Envoy" \cite{envoy_opa}. 
Through these demonstrations, we showcase envoy's ability to be customized to better secure an application.
