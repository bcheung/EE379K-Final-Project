% Describe the overall area of contribution, the crux of the problem,
% and end with highlights of results. For the initial report, end with
% the proposed experiments and what you aim to find out.

Container orchestration technologies like Kubernetes provide a simple way to deploy, scale, 
and manage containerized applications \cite{k8s}. However, as with any production-level 
system, these distributed systems must be properly secured at each appropriate level. In 
this paper, we examine common vulnerabilities of distributed applications and demonstrate 
how we can help mitigate them through Pod Security Policies and network proxies. Pod Security 
Policies are policies that define rules/conditions that a pod must run with in order to 
function within the whole cluster of pods. Network Proxies behave as a gateway between some 
internet user and the network, like a middleman for some network. Proxies, like Envoy, can 
be configured to fortify an application's security by filtering out malicious requests and 
encrypting accepted requests. We first implement Pod Security Policies to improve the base 
security of the application itself by limiting the pod's capabilities and only allowing 
appropriate privileges. By defining the least amount of capabilities required to run the 
application, we reduce the attack surface of the application. Then we implement Envoy with 
Open Policy Agent to reinforce the network security by only accepting specific requests based 
on client privileges. This is achieved by filtering certain packets based on information such 
as the source/destination IP addresses and protocols, which further reduces the attack surface 
and strengthens current security practices. As demonstrated in our project, the additional 
layers of security can help mitigate common vulnerabilities.
