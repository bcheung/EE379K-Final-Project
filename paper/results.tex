First, we will improve the base security of the application through Pod Security Policies.
Then we will add another layer of security through Data Plane filters in order to reduce the
attack surface and prevent suspicious activity. These extra layers of security should hopefully
eliminate the vulnerabilities of the application.

\subsection{Without Pod Security Policies}
Both pods successfully deploy and run in the cluster~\ref{no-psp-k8s-resources}. On the 
\verb|non-root-debian| pod, the user can escalate privilages to run as the \verb|root| user ~\ref{no-psp-priv-esc}. 

\subsection{With Pod Security Policies}
% Explain Pod Security Policies
With a restrictive PSP that requires the user to run as a non-root user, the \verb|non-root-debian| 
pod successfully runs while the \verb|root-debian| pod is prevented from running with a 
\verb|CreateContainerConfigError| as shown in Figure~\ref{fig:psp-k8s-resources} and 
Figure~\ref{fig:psp-describe-root}. The PSP prevents the \verb|root-debian| pod from running because
the policies does not allow the pod to run as the \verb|root| user.

The PSP also prevents privilege escalation as well. The \verb|non-root-debian| pod successfully runs
as a non-root user. The same privelege escalation method is tried with this pod after applying PSP,
and as a result, the privilege escalation fails as shown in Figure~\ref{fig:psp-priv-esc}.

\onecolumn

\begin{figure}[t]
  \begin{center}
    \vspace{-0.2in}
    \psfig{figure=./images/eps/no-psp-k8s-resources.eps,width=7in,height=1.2in} 
    \caption{Screenshot of the Kubernetes resources after deploying the pods without Pod Security Policies}
    \label{fig:no-psp-k8s-resources}
  \end{center}
\end{figure}

\begin{figure}[t]
  \begin{center}
    \vspace{-0.2in}
    \psfig{figure=./images/eps/no-psp-priv-esc.eps,width=7in,height=1.2in} 
    \caption{Screenshot of the privilege escalation on the non-root-debian pod}
    \label{fig:no-psp-priv-esc}
  \end{center}
\end{figure}

\begin{figure}[t]
  \begin{center}
    \vspace{-0.2in}
    \psfig{figure=./images/eps/psp-k8s-resources.eps,width=7in,height=1.2in} 
    \caption{Screenshot of the Kubernetes resources after applying Pod Security Policies and deploying the pods}
    \label{fig:psp-k8s-resources}
  \end{center}
\end{figure}

\begin{figure}[t]
  \begin{center}
    \vspace{-0.2in}
    \psfig{figure=./images/eps/psp-describe-root.eps,width=5.5in,height=2.5in} 
    \caption{Screenshot of the root-debian pod's status after applying the restrictive PSP and deploying}
    \label{fig:psp-describe-root}
  \end{center}
\end{figure}

\begin{figure}[t]
  \begin{center}
    \vspace{-0.2in}
    \psfig{figure=./images/eps/psp-priv-esc.eps,width=7in,height=1.2in} 
    \caption{The non-root-debian pod fails to escalate privileges and login as the root user with the restrictive PSP}
    \label{fig:psp-priv-esc}
  \end{center}
\end{figure}

\twocolumn

\subsection{Network Proxy - Envoy}
We demonstrate just some of the features of Envoy, including the ability to secure HTTP traffic by applying SSL
certificates and redirecting HTTP to HTTPS traffic as well using Envoy's External authorization filter
used with Open Policy Agent to enforce security policies over API requests that the proxy receives.

\subsubsection{SSL Certificate and HTTPS Redirect}
In order to apply SSL certificates to a request and redirect traffic to HTTPS, first a SSL certificate needs to be created.
Ideally, you would generate the certificate from some service, but a simple certificate can be generated using the following command
\begin{minted}{bash}
  mkdir certificates;
  cd certificates;
  openssl req - nodes -new -x509 \
    -keyout "domainname-com.key -out domainname-com.crt \
    -days 365 \
    -subj '/CN=domainname.com/O=Company Name LTD./C=US';
\end{minted}
After creating the SSL certificate, the envoy.yaml file needs to be edited in order to apply a TLS context in the filter.
A TLS context will specify the certificate(s) to be used with the proxy domains. The following lines will apply the TLS context
\begin{minted}{yaml}
  tls_context:
     common_tls_context:
     tls_certificates:
        - certificate_chain:
           filename: "/filepath_to_certificate.crt"
        private_key:
           filname: "/filepath_to_certificate.key"
\end{minted}
After defining the certificates to be used, the proxy will be able to redirect traffic to HTTPS. In order to complete the
redirect, we need to set the https_redirect flag to true.
\begin{minted}{yaml}
  redirect:
     path_redirect: "/"
     https_redirect: true    
\end{minted}
These configurations will enable the envoy proxy to attach a SSL Certificate to network requests as well as redirect any
HTTP traffic to HTTPS.

\subsubsection{Envoy Proxy Configuration with OPA}
In this section, we examine envoy's ability to use external authorization services, specifically checking
if incoming requests have certain authorization privileges. In this example, we have two clients, Brian and Sam, that
want to access an endpoint that has employees. Brian is assigned a guest role, and thus can only get employees but cannot
create an employee. Sam is assigned an admin role, which means Sam can get employees as well as create an employee. This policy
is implemented with Open Policy Agent (OPA). After creating an envoy.yaml configuration file and mapping it to kubernetes,
the OPA policy needs to be created. Below are some code snippets from the policy.rego file
\begin{minted}{rego}
  action_allowed {
    http_request.method == "GET"
    token.payload.role == "guest"
    glob.match("/people", [], http_request.path)
  }
  action_allowed {
    http_request.method == "GET"
    token.payload.role == "admin"
    glob.match("/people", [], http_request.path)
  }
  action_allowed {
    http_request.method == "POST"
    token.payload.role == "admin"
    glob.match("/people", [], http_request.path)
    lower(input.parsed_body.firstname) != base64url.decode(token.payload.sub)
  }
\end{minted}
This defines the actions that are allowed. As shown, the admin role is allowed to execute GET and POST requests to the "/people" endpoint, while
a guest can only execute a GET request. Below, there are screenshots of Brian, our guest, executing both GET and POST requests, and the GET request
is allowed while the POST request is denied. The same requests are executed by Sam, our admin, and both requests are allowed.

\begin{figure}[t]
  \begin{center}
    \vspace{-0.2in}
    \psfig{figure=./images/eps/Brian_OPA.eps,width=7in,height=1.2in} 
    \caption{Guest role executing both GET and POST requests}
    \label{fig:opa-guest}
  \end{center}
\end{figure}

\begin{figure}[t]
  \begin{center}
    \vspace{-0.2in}
    \psfig{figure=./images/eps/Sam_OPA.eps,width=7in,height=1.2in} 
    \caption{Admin role executing both GET and POST requests}
    \label{fig:opa-admin}
  \end{center}
\end{figure}
