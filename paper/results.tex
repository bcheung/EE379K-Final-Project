First, we will improve the base security of the application through Pod Security Policies.
Then we will add another layer of security through Data Plane filters in order to reduce the
attack surface and prevent suspicious activity. These extra layers of security should hopefully
eliminate the vulnerabilities of the application.

\subsection{Without Pod Security Policies}
Both pods successfully deploy and run in the cluster~\ref{no-psp-k8s-resources}. On the 
\verb|non-root-debian| pod, the user can escalate privilages to run as the \verb|root| user ~\ref{no-psp-priv-esc}. 

\subsection{With Pod Security Policies}
% Explain Pod Security Policies
With a restrictive PSP that requires the user to run as a non-root user, the \verb|non-root-debian| 
pod successfully runs while the \verb|root-debian| pod is prevented from running with a 
\verb|CreateContainerConfigError| as shown in Figure~\ref{fig:psp-k8s-resources} and 
Figure~\ref{fig:psp-describe-root}. The PSP prevents the \verb|root-debian| pod from running because
the policies does not allow the pod to run as the \verb|root| user.

The PSP also prevents privilege escalation as well. The \verb|non-root-debian| pod successfully runs
as a non-root user. The same privelege escalation method is tried with this pod after applying PSP,
and as a result, the privilege escalation fails as shown in Figure~\ref{fig:psp-priv-esc}.

\onecolumn

\begin{figure}[t]
  \begin{center}
    \vspace{-0.2in}
    \psfig{figure=./images/eps/no-psp-k8s-resources.eps,width=7in,height=1.2in} 
    \caption{Screenshot of the Kubernetes resources after deploying the pods without Pod Security Policies}
    \label{fig:no-psp-k8s-resources}
  \end{center}
\end{figure}

\begin{figure}[t]
  \begin{center}
    \vspace{-0.2in}
    \psfig{figure=./images/eps/no-psp-priv-esc.eps,width=7in,height=1.2in} 
    \caption{Screenshot of the privilege escalation on the non-root-debian pod}
    \label{fig:no-psp-priv-esc}
  \end{center}
\end{figure}

\begin{figure}[t]
  \begin{center}
    \vspace{-0.2in}
    \psfig{figure=./images/eps/psp-k8s-resources.eps,width=7in,height=1.2in} 
    \caption{Screenshot of the Kubernetes resources after applying Pod Security Policies and deploying the pods}
    \label{fig:psp-k8s-resources}
  \end{center}
\end{figure}

\begin{figure}[t]
  \begin{center}
    \vspace{-0.2in}
    \psfig{figure=./images/eps/psp-describe-root.eps,width=5.5in,height=2.5in} 
    \caption{Screenshot of the root-debian pod's status after applying the restrictive PSP and deploying}
    \label{fig:psp-describe-root}
  \end{center}
\end{figure}

\begin{figure}[t]
  \begin{center}
    \vspace{-0.2in}
    \psfig{figure=./images/eps/psp-priv-esc.eps,width=7in,height=1.2in} 
    \caption{The non-root-debian pod fails to escalate privileges and login as the root user with the restrictive PSP}
    \label{fig:psp-priv-esc}
  \end{center}
\end{figure}

\twocolumn

\subsection{With Envoy}
Envoy is to be used alongside the application that we are testing and
will be used to monitor the inflow and outflow of traffic. With envoy,
we will be able to see how well the implemented pod security policies
are working.
