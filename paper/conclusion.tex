While Pod Security Policies and Network Proxies have similar goals of maintaining/strengthening 
the strength of the security of an application, they do have different use cases and have unique 
strengths when it comes to application/network security. Pod Security Policies secure the base 
application and the Kubernetes cluster that the application is hosted on. On the other hand, a 
network proxy is the middleman between the client and the network and thus is met with every network 
request. Furthermore, designing a network proxy that can effectively handle these network requests, 
whether that be filtering out malicious requests, or assigning security policies onto API requests 
is extremely important in protecting the overall security integrity of the application as well as 
reducing the load that the rest of the application has to deal with. Network proxies can be viewed 
as the outer line of defense, while Pod Security Policies act as the base-line defense that protects 
the system if a threat manages to bypass the network proxy. Both services add a layer of security 
that helps mitigate threats at different levels of the application, which ultimately minimizes the 
attack surface of the application.
