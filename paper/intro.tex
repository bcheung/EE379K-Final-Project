
% Second para: describe the key problem that if solved would make
% an impact. Why the current approaches leave a gap?

% Third: describe your approach. Key insight that enables your approach,
% and what is novel/interesting about the insight.

% Fourth, fifth: Delve deeper into the approach and experimental setup.
% In the final report, describe key findings. 
As container orchestration technologies and cloud platforms like Kubernetes and Docker Hub simplify 
the deployment and scaling process and increase the accessibility of open-sourced software, developers 
must be cautious of unintentional vulnerabilities when deploying large-scale distributed applications. 
Common vulnerabilities of these distributed applications can stem from misconfiguration and misuse of 
legitimate privileges, unknown vulnerabilities (often from open-source software), and an overall lack 
of security. These vulnerabilities may allow attackers to gain privileged control of the system. This 
issue presents the need for preventative measures that can be implemented with Kubernetes' Pod Security 
Policies. 

Kubernetes provides a framework that allows developers to create Pod Security Policies (PSP) which 
defines the rules that pods in a cluster must follow in order to be allowed to operate. This ensures 
that pods can and will run only if it maintains the correct privileges and resources defined in the PSPs. 
When a pod is deployed, the PSP acts as a  gatekeeper that will compare the pod security configuration to 
what is defined  in the PSP. Some examples of how PSPs can limit pod behaviors include, preventing 
privilege escalation, restricting pods from accessing host namespaces, filesystem, and networks, restrict 
the amount of user/groups that a pod can run, and more.

Systems that involve networking may need additional layers of security in order to further reduce the 
attack surface. Common threats and vulnerabilities in networks can include denial of service attacks 
and snooping to extract confidential and private information \cite{mns}. These threats can be mitigated 
through network filtering, which is the practice of monitoring the inflow/outflow of packets in a network. 

There are two types of filtering, ingress and egress filtering. Ingress filtering is the technique of 
monitoring incoming packet data. It is considered the first-line of defense in a network because it blocks 
out unwanted inflow traffic to the network. While this isn't a robust and complete form of defense, it's 
beneficial because it can greatly reduce the load on some proxy or firewall, and it's an effective for 
getting rid of the majority of unwanted traffic. Egress filtering is the technique of monitoring the flow 
of outbound network traffic and prevents any outbound connections to potential threats/unwanted hosts. 
Egress filtering can be used to disrupt malware, block unwanted services, and gives greater awareness of 
network traffic.

For this experiment, we implemented two separate distributed applications with Kubernetes to demonstrate the 
possible ways Pod Security Policies and network proxies can mitigate common vulnerabilities and threats. 
The first part demonstrates how Pod Security Policies can prevents privilege escalation on distributed 
applications deployed with Kubernetes. The second part demonstrates the effectiveness and use case of an 
open source network proxy, called Envoy, that is used alongside a Kubernetes deployment. 

% End with outline or what comes next and why.
Through our paper, we will show the effectiveness of and the need to implement robust Pod Security Policies
as well as a Network Proxy to aide with network filtering, in order to improve the overall security of a
distributed application. 
