%--------------------------------------------
\begin{figure}[t]
\begin{center}
\vspace{-0.2in}
%%\psfig{figure=1bcounter.eps,width=2.5in,height=1.8in} 
\psfig{figure=1bcounter.eps,width=1.7in,height=1.2in} 
\caption{A 1-bit counter with reset. With the conventional technique of OR-ing all input shadow values, the feedback loop ensures that a 
counter shall never be trusted once it gets marked as untrusted. Our shadow logic is more precise and recognizes that a trusted reset 
guarantees a trusted $0$ in the counter value.}
\label{fig:1bcounter}
\end{center}
\end{figure}
%--------------------------------------------

Motivation describes the most important of the related works. The ones that 
you either build on, prove/disprove, or in any way ``extend''. 

Other related work, that is orthogonal to your approach but is in the same
general problem-area, can be included in a separate related work section.
One good place for that is at the end, so it doesn't disrupt the story here.

It is important to continue researching how network filtering can lead to better application security, especially with a lot of technologies and applications all moving into the cloud.
A strong initial filter that effectively identifies a majority of attacks dramatically decreases the chance of a successful attack. If most of the attempted attacks are initially thwarted,
there are few attacks that can get through the security in place. A good network filter means less attacks an application has to be wary of, which should lead to a higher defense rate.