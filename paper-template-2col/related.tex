\verb|https://www.researchgate.net/publication/324562008_Threats_and_Vulnerabilities_of_Cloud_Computing_A_Review|
The article talks about how more and more enterprises are moving their workloads onto the cloud and while
security has evolved over time, is still a major concern. The paper goes into details about the various forms
of threats and vulnerabilities of the cloud, specifically listing and detailing 17 threats. This paper
provides a good foundation for understanding common threats that exploit cloud vulnerabilities.

\verb|https://www.researchgate.net/publication/267691532_MODERN_NETWORK_SECURITY_ISSUES_AND_CHALLENGES|
This paper discusses the threats that networks face and the current network security practices to counteract these attacks.
The paper begins by detailing security attacks, security measures, and security tools. The paper goes into great detail about
different security methods, such as application gateways and packet filtering. The paper discusses different things that organizations
can do to prepare for these attacks and the various technology options.

\verb|https://www.researchgate.net/publication/289756317_Security_Threats_on_Cloud_Computing_Vulnerabilities|
This paper further discusses the vulnerabilities of cloud computing services. The paper details cloud service models and talks about
the 3 layers of cloud computing: system layer (IaaS), platform layer (PaaS), and application layer (SaaS). The paper then analyzes
the various security issues that each layer faces and talks about the threats that exploit those vulnerabilities.

\verb|https://www.researchgate.net/publication/334548954_Cloud_Security|
This paper discusses the advantages of using cloud services and also reveals the dangers and risks of those services.

\verb|https://nvd.nist.gov/vuln/detail/CVE-2019-5736|
This is a known vulnerability, CVE-2019-5736~\cite{cve}, that allows attackers to execute commands as root within two types of containers,
a new container with an attack-controlled image and an already existing container that an attacker has had access to in the past.

\verb|https://cyber-defense.sans.org/resources/papers/gsec/packet-filter-basic-network-security-tool-100197|
This article goes into detail about what packet filtering is and how it is used as a network security tool.
The paper details the benefits of packet filters and gives a simple implementation of it and discusses
the limitations of packet filtering.
